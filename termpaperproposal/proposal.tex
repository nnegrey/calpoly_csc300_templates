% Term paper proposal template - Ilona Sparks
% CSC 300: Professional Responsibilities
% Dr. Clark Turner

% One Column Format
\documentclass[12pt]{article}

% Allows the user to set up multiple columns or just one column
\usepackage{multicol}
\usepackage{lipsum}

\usepackage{setspace}
\usepackage{url}

%%% PAGE DIMENSIONS
\usepackage[top=2cm, bottom=2.54cm, left=2.54cm, right=2.54cm]
{geometry}\usepackage{geometry}
\geometry{letterpaper}

%%% TITLE PAGE %%%
\begin{document}

\title{\vfill Term Paper Proposal Template} %\vfill gives us the black space at the top of the page
\author{
Noah Negrey \vspace{10pt} \\
\small CSC 300: Professional Responsibilities \\
\small Dr. Clark Turner \\
}
\date{October 6, 2014} %Or use \today for today's Date

\maketitle

\vfill  %in combination with \newpage this forces the abstract to the bottom of the page
\begin{abstract}
One or two paragraphs to describe in very general terms the motivating facts, the question asked, one or two arguments and your ultimate answer and the basic principles upon which it rests. This would be the 30 second summary you might give your mother or friend. % \cite{handout}
\end{abstract}

\thispagestyle{empty} %remove page number from title page, but still keep it as pg #1
\newpage
% ****************************************************************************************************************************************

%%% TABLE OF CONTENTS %%%

%Create a table of contents with all headings of level 3 and above.
%http://en.wikibooks.org/wiki/LaTeX/Document_Structure#Table_of_contents has
%info on customizing the table of contents
\tableofcontents

\thispagestyle{empty}  %Remove page number from TOC
\newpage
% ****************************************************************************************************************************************

% Setup a 2 column format for the proposal
\begin{multicols}{2}

% Set the first page with content as page 1
\setcounter{page}{1}


%%% KNOWN FACTS  %%%
\section{Facts}

Known facts that are not disputed that lead to your question. Do not judge these facts or make anything like an argument for an answer in here. Just note the facts that give us the general background and end them with the facts leading to the controversy you are interested in. The reader should naturally be asking the question you'll be asking by that point in your paper. In general, attach your facts to a specific case, the more specific and detailed the facts, the better for your analysis. Cite all facts to their sources. % \cite{handout}

%%% RESEARCH QUESTION %%%
\section{Research Question}

Your research question -- this is the ethical question you are interested in answering. It should be one simple sentence and lead to a yes/no answer. It needs to be very narrowly focused, specific, and not abstract at all. It's best to question a detailed case in the general area of your interest. Open ended questions are very hard to answer. \cite{handout}

%%% EXTANT ARGUMENTS FROM EXTERNAL SOURCES %%%
% Arguments For
\section{Arguments}

Extant arguments -- this is where you gather the arguments made by others interested in the same question. No judgments, just repeat their arguments for the answer in the best possible light from the arguer's perspective. Cover both sides of your question (the ``yes'' side and the ``no'' side) to get a complete picture of how others are thinking about it. Do not include any general ethical principles in here unless they are explicitly written up in the arguments. Cite all arguments to their sources. \cite{handout}

\subsection{Argument For}
Here forth are the arguments for your case.
\subsubsection{The first argument for your topic}
Lorem ipsum dolor sit amet, consectetuer adipiscing elit. Ut purus elit, vestibulum ut, placerat ac, adipiscing vitae, felis. Curabitur dictum gravida mauris. Nam arcu libero, nonummy eget, consectetuer id, vulputate a, magna. Donec vehicula augue eu neque. Pel- lentesque habitant morbi tristique senectus et netus et malesuada fames ac turpis egestas. Mauris ut leo.
\subsubsection{Argument Number 2}
Lorem ipsum dolor sit amet, consectetuer adipiscing elit. Ut purus elit, vestibulum ut, placerat ac, adipiscing vitae, felis. Curabitur dictum gravida mauris. Nam arcu libero, nonummy eget, consectetuer id, vulputate a, magna. Donec vehicula augue eu neque. Pel- lentesque habitant morbi tristique senectus et netus et malesuada fames ac turpis egestas. Mauris ut leo.
% Arguments Against
\subsection{Arguments Against}
Here forth are the arguments against your case.
\subsubsection{Argument Number 1}
Lorem ipsum dolor sit amet, consectetuer adipiscing elit. Ut purus elit, vestibulum ut, placerat ac, adipiscing vitae, felis. Curabitur dictum gravida mauris. Nam arcu libero, nonummy eget, consectetuer id, vulputate a, magna. Donec vehicula augue eu neque. Pel- lentesque habitant morbi tristique senectus et netus et malesuada fames ac turpis egestas. Mauris ut leo.
\subsubsection{Argument Number 2}
Lorem ipsum dolor sit amet, consectetuer adipiscing elit. Ut purus elit, vestibulum ut, placerat ac, adipiscing vitae, felis. Curabitur dictum gravida mauris. Nam arcu libero, nonummy eget, consectetuer id, vulputate a, magna. Donec vehicula augue eu neque. Pel- lentesque habitant morbi tristique senectus et netus et malesuada fames ac turpis egestas. Mauris ut leo.
%%% ANALYSIS %%%
\section{Analysis}
Applicable analytic principles -- give a list of the basic ethical (and other) principles you'll rely on to come up with your analysis, include several explicit principles from the SE Code of Ethics, deontological principles, utilitarianism (rule-utilitarianism) as well as others that will aid you. Indicate generally how they apply to your specific case. Cite any additional facts or principles you'll need. Cite to sources for the principles you list.  \cite{handout}

%%% CONCLUSION %%%
\section{Conclusion}
Give a short abstract of the basics you expect to analyze and present in your paper. Divide it into sections that make sense for your work.

One way would be to: a) start with deontological perspectives as a section where you analyze those arguments based on the inherent ethics of the act itself rather than the results or trade-offs; then, b) use a utilitarian perspective and list the appropriate analyses of the trade-offs and stakeholders to define the most desired results and how to get them. Be explicit about the trade-offs (what value is balanced against what other value, which stakeholders win, which stakeholders lose...) What is the ``utility'' in ``utilitarian'' in your case -- what value do you want to advance the most (derived from the general utilitarian ``happiness'')? How do you maximize (or optimize) it?

Note that the SE Code should be the center of your ethical analysis (and remember that it includes both deontological and utilitarian [and more] principles you can utilize). Estimate where you'll end up for your answer (you can change your mind in the final paper!). Keep referencing sources for any additional facts, quotes, or other information you might use here.  \cite{handout}

%end the two column format
\end{multicols}
\newpage
% ****************************************************************************************************************************************

%%% BIBLIOGRAPHY %%%
%cite all the references from the bibtex you haven't explicitly cited
\nocite{*}
\bibliographystyle{IEEEannot}
\bibliography{proposal}
\newpage
% ****************************************************************************************************************************************
\end{document}
